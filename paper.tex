\documentclass{report}

\usepackage{amsmath}
\usepackage{geometry}
\usepackage{hyperref}
\usepackage{listings}
\geometry{
    a4paper,
}
\hypersetup{
    pdftitle={Perceptual Convolution},
    pdfauthor={Tobias Hienzsch},
}

\title{Perceptual Convolution}
\author{Tobias Hienzsch}
\date{\today}

\begin{document}

\pagenumbering{gobble}
\maketitle
\newpage

\tableofcontents
\newpage

\thispagestyle{empty}
\listoffigures
\listoftables
\newpage

\pagenumbering{arabic}

\chapter{Dynamic Range}

Dynamic range describes the ratio of the amplitude of the loudest possible
undistorted signal to the noise floor. The maximum achievable dynamic range for
a digital audio system with Q-bit uniform quantization:
\footnote{\url{https://en.wikipedia.org/wiki/Dynamic_range}}

\begin{equation}
    \begin{split}
        DR_{analog}  & = 20\log_{10} \left(\frac{V_{ceiling}}{V_{floor}}\right) \\
        DR_{digital} & = 20\log_{10} \left(\frac{2^Q}{1}\right) \\
                     & = 6.02 * Q
    \end{split}
\end{equation}

\begin{center}

\end{center}

\begin{table}[h!]
    \centering
    \begin{tabular}{||c c c||}
        \hline
        Bit-Depth & Dynamic-Range (dB) & Example Application \\
        \hline
        \hline
        8 & 48.16 & Legacy Hardware \\
        \hline
        12 & 72.24 & Legacy Hardware \\
        \hline
        16 & 96.32 & Audio-CD \\
        \hline
        24 & 144.49 & High-quality ADC/DAC \\
        \hline
        32 & 192.65 & Digital Audio Workstation (DAW) \\
        \hline
        64 & 385.31 & Scientific Computing \\
        \hline
    \end{tabular}
    \caption{Dynamic-Range for given Bit-Depth}
    \label{table:1}
\end{table}

\section{Crest Factor}


When choosing quantization for any signal $x[t]$ your SNR for the quantization
noise is determined by the crest factor of the signal which is the peak divided
by the RMS value, i.e.

\begin{equation}
    C_x = \frac{x_{peak}}{x_{rms}}
\end{equation}

If you quantize with N bits at the clipping level, your quantization step, will
be (assuming signed signals)

\begin{equation}
    \Delta_q = \frac{x_{peak}}{2^{N-1}}
\end{equation}

and the quantization noise in dB is

\begin{equation}
    \begin{split}
        L_q & = 20\log_{10} \frac{\Delta_q}{\sqrt{12}} \\
            & = 20\log_{10} \frac{x_{peak}}{2^{N-1}\sqrt{12}}
    \end{split}
\end{equation}

and you SNR becomes

\begin{equation}
    \begin{split}
        SNR & = 20\log_{10} x_{rms} - L_q \\
            & = 20\log_{10} 2^{N-1}\sqrt{12}\frac{x_{rms}}{x_{peak}} \\
            & = 20\log_{10} \frac{2^{N-1}\sqrt{12}}{C_x}
    \end{split}
\end{equation}

So it's inversely proportional to the Crest factor. Most audio signals have a
"moderate" Crest factor in the time domain, maybe 15 dB or so, which gives you
an SNR of 86dB or thereabouts for 16-bit quantization. Not great, but not
terrible either.

However the Crest factor in the frequency domain can be much higher. An extreme
case is a full scale sine wave, which has a crest factor of roughy

\begin{equation}
    C_{sine,frequency} = \sqrt{\frac{M_{FFT}}{2}} \tag{5}
\end{equation}

Where $M_{FFT}$ is the FFT length. For an FFT length of 2048 and 16 bit
quantization this comes out to be a whopping 30 dB, reduces your SNR to the
70 dB range.

You have to decide whether this is acceptable for your application or you have
to deploy some dynamic stage scaling scheme in your FFT.

See full conversation on dsp.stackexchange.com
\footnote{\url{https://dsp.stackexchange.com/questions/91811}}

\chapter{Sample Data-Type}

\section{Integer}

\section{Fixed-Point}

\subsection{Arithmetic}

\begin{lstlisting}[language=C++]

constexpr int16_t Q = 1;
constexpr int16_t K = 1 << (Q - 1);

int16_t sat16(int32_t x)
{
	if (x > 0x7FFF) return 0x7FFF;
	else if (x < -0x8000) return -0x8000;
	else return (int16_t)x;
}

int16_t add(int16_t x, int16_t y) {
    return x + y
}
int16_t sub(int16_t x, int16_t y) {
    return x - y
}
int16_t mul(int16_t x, int16_t y) {
    int32_t tmp = (int32_t)(x) * (int32_t)(y);
    tmp += K;
    return sat16(tmp >> Q);
}
\end{lstlisting}

\subsection{SIMD Support}

\section{Floating-Point}

\end{document}
